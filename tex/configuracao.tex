\section{Introdução}

O Plano de Testes do Sistema Automatizado de Emissão de Certificados Digitais ICPEdu descreve os itens a serem testados. Este capítulo delimitará as funcionalidades que entrarão no plano de testes, relacionando-as. A metodologia dos testes também será apresentada.

\section{Itens de teste}

Serão testadas as funcionalidades requeridas ao Sistema. Com o teste dos requisitos funcionais espera-se que todos os módulos funcionais sejam verificados.

\section{Funcionalidades a serem testadas}

Os testes serão realizados de forma a abranger todas as funcionalidades do sistema.

\section{Abordagem}

Os testes de software serão aplicados de acordo com o seguinte padrão:

\begin{enumerate}
 \item Será utilizada a versão atual existente do software
 \item Os testes podem ser realizados nas máquinas dos responsáveis pelos testes, desde que estes sigam os requisitos de sistema operacionais necessários à execução do serviço (Ubuntu 14.04)
 \item Os testes devem ser executados em todos os navegadores, no Windows e Linux (se possível), dando preferência sempre ao MoziFirefox pois é o navegador padrão utilizado pela RNP.
 \item Será executado cada item de cada caso de teste e reportado de acordo com o modelo de Entradas e Saídas apresentado neste plano;
 \item O plano de testes que segue estará dividido em duas grandes partes: Funcionalidades que dizem respeito ao Administrador do sistema, e o funcionalidades que dizem respeito aos Operadores do sistema.
\end{enumerate}

A compilação e execução do projeto como um todo deverá seguir o tutorial desenvolvido pelo responsável pelo projeto, ou seja, o sistema deve executar as funcionalidades e responder corretamente, conforme previsto, para todas as entradas inseridas em cada funcionalidade.

\section{Critério de aceitação de item}

Os itens dos casos de testes apresentados ao longo deste Plano de Testes deverão apresentar exatamente as saídas determinadas para as possíveis entradas informadas. Caso algum item demonstre qualquer saída não esperada, deverá ser informado para as partes de interesse. Não há um prazo pré-determinado para obter e reportar as saídas encontradas.

\section{Artefatos de teste a serem entregues}

O resultado de cada aplicação dos itens dos Casos de Teste será um relatório de acordo com o modelo descrito na seção 5 deste plano. Todos os itens aplicados deverão ter uma entrada no relatório, sendo ele realizado com sucesso ou com falha.

\section{Tarefas de teste}

As tarefas dos testes deverão ser realizadas a partir do estado do software indicado em cada teste, fornecer como entrada os dados informados e checar se a saída da aplicação ocorre como a indicada.
Todos os itens testados deverão ser colocados no relatório cujo modelo é apresentado na seção 5. Os itens que não estiverem em conformidade com o esperado deverão descrever o resultado apresentado.

\section{Necessidades de ambiente}

É necessário que seja utilizado sistema operacional linux, com servidor web apache, php versão 4 ou superior, banco de dados postgresql e é necessária a instalação da biblioteca libcryptosec.

\section{Responsabilidades}

A responsabilidade de modelar, como também de aplicar, todos os Casos de Testes no Sistema será do gerente do projeto e de quem ele julgar necessário delegar a atividade.
Os defeitos de software encontrados durante a execução deste plano de teste serão reportados diretamente ao gerente do projeto para que tais erros possam ser concertados com prioridade.

\section{Necessidades de treinamento}

Quaisquer dúvidas quanto às funcionalidades do sistema, e os conceitos encontrados no mesmo, podem ser esclarecidas utilizando os outros documentos que suportam o sistema, como Manual do Usuário e FAQ (Frequently Asked Questions). Por este motivo, não se faz necessário um treinamento específico.

\section{Cronograma}

Não se aplica, pois cada testador pode criar seu próprio cronograma para a execução dos testes. 